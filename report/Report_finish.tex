\documentclass[12pt,a4paper]{report}
\usepackage[utf8]{inputenc}
\usepackage[russian]{babel}
\usepackage[OT1]{fontenc}
\usepackage{amsmath}
\usepackage{amsfonts}
\usepackage{amssymb}
\usepackage{graphicx}
\usepackage{cmap}					% поиск в PDF
\usepackage{mathtext} 				% русские буквы в формулах
%\usepackage{tikz-uml}               % uml диаграммы

% TODOs
\usepackage[%
  colorinlistoftodos,
  shadow
]{todonotes}

% Генератор текста
\usepackage{blindtext}%quadEquation.c
%\lstinputlisting[]
%{../subdirproject/lib/quadEquation.c}

%\vspace{\baselineskip}

%quadEquationUI.c
%\lstinputlisting[]
%{../sources/subdirproject/app/quadEquationUI.c}


%------------------------------------------------------------------------------

% Подсветка синтаксиса
\usepackage{color}
\usepackage{xcolor}
\usepackage{listings}
 
 % Цвета для кода
\definecolor{string}{HTML}{B40000} % цвет строк в коде
\definecolor{comment}{HTML}{008000} % цвет комментариев в коде
\definecolor{keyword}{HTML}{1A00FF} % цвет ключевых слов в коде
\definecolor{morecomment}{HTML}{8000FF} % цвет include и других элементов в коде
\definecolor{captiontext}{HTML}{FFFFFF} % цвет текста заголовка в коде
\definecolor{captionbk}{HTML}{999999} % цвет фона заголовка в коде
\definecolor{bk}{HTML}{FFFFFF} % цвет фона в коде
\definecolor{frame}{HTML}{999999} % цвет рамки в коде
\definecolor{brackets}{HTML}{B40000} % цвет скобок в коде
 
 % Настройки отображения кода
\lstset{
language=C, % Язык кода по умолчанию
morekeywords={*,...}, % если хотите добавить ключевые слова, то добавляйте
 % Цвета
keywordstyle=\color{keyword}\ttfamily\bfseries,
stringstyle=\color{string}\ttfamily,
commentstyle=\color{comment}\ttfamily\itshape,
morecomment=[l][\color{morecomment}]{\#}, 
 % Настройки отображения     
breaklines=true, % Перенос длинных строк
basicstyle=\ttfamily\footnotesize, % Шрифт для отображения кода
backgroundcolor=\color{bk}, % Цвет фона кода
%frame=lrb,xleftmargin=\fboxsep,xrightmargin=-\fboxsep, % Рамка, подогнанная к заголовку
frame=tblr
rulecolor=\color{frame}, % Цвет рамки
tabsize=3, % Размер табуляции в пробелах
showstringspaces=false,
 % Настройка отображения номеров строк. Если не нужно, то удалите весь блок
numbers=left, % Слева отображаются номера строк
stepnumber=1, % Каждую строку нумеровать
numbersep=5pt, % Отступ от кода 
numberstyle=\small\color{black}, % Стиль написания номеров строк
 % Для отображения русского языка
extendedchars=true,
literate={Ö}{{\"O}}1
  {Ä}{{\"A}}1
  {Ü}{{\"U}}1
  {ß}{{\ss}}1
  {ü}{{\"u}}1
  {ä}{{\"a}}1
  {ö}{{\"o}}1
  {~}{{\textasciitilde}}1
  {а}{{\selectfont\char224}}1
  {б}{{\selectfont\char225}}1
  {в}{{\selectfont\char226}}1
  {г}{{\selectfont\char227}}1
  {д}{{\selectfont\char228}}1
  {е}{{\selectfont\char229}}1
  {ё}{{\"e}}1
  {ж}{{\selectfont\char230}}1
  {з}{{\selectfont\char231}}1
  {и}{{\selectfont\char232}}1
  {й}{{\selectfont\char233}}1
  {к}{{\selectfont\char234}}1
  {л}{{\selectfont\char235}}1
  {м}{{\selectfont\char236}}1
  {н}{{\selectfont\char237}}1
  {о}{{\selectfont\char238}}1
  {п}{{\selectfont\char239}}1
  {р}{{\selectfont\char240}}1
  {с}{{\selectfont\char241}}1
  {т}{{\selectfont\char242}}1
  {у}{{\selectfont\char243}}1
  {ф}{{\selectfont\char244}}1
  {х}{{\selectfont\char245}}1
  {ц}{{\selectfont\char246}}1
  {ч}{{\selectfont\char247}}1
  {ш}{{\selectfont\char248}}1
  {щ}{{\selectfont\char249}}1
  {ъ}{{\selectfont\char250}}1
  {ы}{{\selectfont\char251}}1
  {ь}{{\selectfont\char252}}1
  {э}{{\selectfont\char253}}1
  {ю}{{\selectfont\char254}}1
  {я}{{\selectfont\char255}}1
  {А}{{\selectfont\char192}}1
  {Б}{{\selectfont\char193}}1
  {В}{{\selectfont\char194}}1
  {Г}{{\selectfont\char195}}1
  {Д}{{\selectfont\char196}}1
  {Е}{{\selectfont\char197}}1
  {Ё}{{\"E}}1
  {Ж}{{\selectfont\char198}}1
  {З}{{\selectfont\char199}}1
  {И}{{\selectfont\char200}}1
  {Й}{{\selectfont\char201}}1
  {К}{{\selectfont\char202}}1
  {Л}{{\selectfont\char203}}1
  {М}{{\selectfont\char204}}1
  {Н}{{\selectfont\char205}}1
  {О}{{\selectfont\char206}}1
  {П}{{\selectfont\char207}}1
  {Р}{{\selectfont\char208}}1
  {С}{{\selectfont\char209}}1
  {Т}{{\selectfont\char210}}1
  {У}{{\selectfont\char211}}1
  {Ф}{{\selectfont\char212}}1
  {Х}{{\selectfont\char213}}1
  {Ц}{{\selectfont\char214}}1
  {Ч}{{\selectfont\char215}}1
  {Ш}{{\selectfont\char216}}1
  {Щ}{{\selectfont\char217}}1
  {Ъ}{{\selectfont\char218}}1
  {Ы}{{\selectfont\char219}}1
  {Ь}{{\selectfont\char220}}1
  {Э}{{\selectfont\char221}}1
  {Ю}{{\selectfont\char222}}1
  {Я}{{\selectfont\char223}}1
  {і}{{\selectfont\char105}}1
  {ї}{{\selectfont\char168}}1
  {є}{{\selectfont\char185}}1
  {ґ}{{\selectfont\char160}}1
  {І}{{\selectfont\char73}}1
  {Ї}{{\selectfont\char136}}1
  {Є}{{\selectfont\char153}}1
  {Ґ}{{\selectfont\char128}}1
  {\{}{{{\color{brackets}\{}}}1 % Цвет скобок {
  {\}}{{{\color{brackets}\}}}}1 % Цвет скобок }
}
 
 % Для настройки заголовка кода
\usepackage{caption}
\DeclareCaptionFont{white}{\color{сaptiontext}}
\DeclareCaptionFormat{listing}{\parbox{\linewidth}{\colorbox{сaptionbk}{\parbox{\linewidth}{#1#2#3}}\vskip-4pt}}
\captionsetup[lstlisting]{format=listing,labelfont=white,textfont=white}
\renewcommand{\lstlistingname}{Код} % Переименование Listings в нужное именование структуры


%------------------------------------------------------------------------------
\author{А.~А.~Ильин}
\title{Программирование}
\begin{document}
\maketitle
\chapter{Основные конструкции языка}
%############################################################
\section{Задание 1}
\subsection{Задание}

Пользователь задает угол в градусах, минутах и секундах. Вывести значение того же угла в радианах.

\subsection{Теоретические сведения}

Радиан - радианная мера угла. Радиан свазан с градусами следующим соответствием:

\begin{displaymath}1 radian = 180 / \pi degrees \end{displaymath}
\
Градусы в свою очередь делятся на секунды и минуты:
 \begin{math}1 degree = 60 min \end{math} 
 \begin{math}1 min = 60 sec \end{math} 

Для реализации данного алгоритма были использованы функции стандартной библиотеки, прототипы которых находятся в файле stdio.h для ввода и вывода информации и math.h для выполнения необходимых вычислений.


\subsection{Проектирование}
Для более удобного хранения данных, а так же их передачи была использована структура Angle. Она содержит 3 поля, которые должны содержать целые числа, соответсвующие градусам, минутам и секундам.

В ходе проектирования было решено выделить одну функцию:

	\begin{itemize}
	

		\item void translation(double, Angle*)
		
		 Функция вычисляет переводит из радиан в градусы.
		 Параметрами функциия являются переменная типа double и указатель на созданную структуру. Первое значение соответствует числу радиан, переданному пользователем, а в структуру на которую получен указетль будет записаны градусы, минуты и секунды. 
		 				
	\end{itemize}

\subsection{Описание тестового стенда и методики тестирования}
Среда разработки QtCreator 3.5.0, компилятор gcc (Debian 4.9.2-10) 4.9.2, операционная система Linux version 3.16.0-4-586.

Для тестирования работы программы были выполнены статический анализ, также было проведено автоматической тестирование.

\subsection{Тестовый план и результаты тестирования}
		Для статического анализа использовалась утилита Cppcheck.
		
		\vspace{\baselineskip}
		 Cppcheck выдал незначительные предупреждения.
		
		\vspace{\baselineskip}
		При автоматическом тестировании вызывалась функция, затем полученные значения сравнивались с ожидаемыми значениями. Результаты тестирования представлены в листингах.
		
		\vspace{\baselineskip}
	 

 
\subsection{Выводы}

При выполнении задания я научился работать со структурами, отработал свои навыки в работе с основными конструкциями языка и получил опыт в организации функций одной программы.

\subsection*{Листинги}

translation.c
\lstinputlisting[]
{../subdirprojectt/lib/translation.c}

\vspace{\baselineskip}

quadEquationUI.c
\lstinputlisting[]
{../subdirprojectt/app/ui_translation.c}


\section{Задание 2}
\subsection{Задание}

Мой возраст. Для заданного $N$ рассматриваемого как возраст человека, вывести фразу вида: «Мне 21 год», «Мне 32 года», «Мне 12 лет». 

\subsection{Теоритические сведения}

В ходе выполения задания для произведения необходимых вычислений и преобразований использовались перечисляемый тип и деление с остатком "\%". Также использовалась конструкция if...else. Кроме того, были применены функции стандартной библотеки из заголовочного файла stdio.h для ввода и вывода информации.

\subsection{Проектирование}


В ходе проектирования были выделены следущая функция:

\begin{itemize}

	\item int tell\_ me\_ age(int)

	Функция проверяет число на несколько условий и возвращает один из идентефикаторов. Перечилсяемый тип был использован для систематизации вывода информации.
	
\end{itemize}


\subsection{Описание тестового стенда и методики тестирования}
Среда разработки QtCreator 3.5.0, компилятор gcc (Debian 4.9.2-10) 4.9.2, операционная система Linux version 3.16.0-4-586.

Для тестирования работы программы был выполнены статический анализ.

\subsection{Тестовый план и результаты тестирования}

		Для статического анализа использовалась утилита Cppcheck.
		
		\vspace{\baselineskip}
		
 Cppcheck выдал незначительные предупреждения.		
\subsection{Выводы}

При выполнении задания я получил опыт в организации функций одной программы.

\subsection*{Листинги}
tell\_ me\_ age.c
\lstinputlisting[]
{../subdirprojectt/lib/tell_me_age.c}

\vspace{\baselineskip}

ui\_ tell\_ me\_ age.c
\lstinputlisting[]
{../subdirprojectt/app/ui_tell_me_age.c}


\chapter{Циклы}
\section{Задание 1}
\subsection{Задание}

Найти число, полученное из данного дублированием четных цифр.

\subsection{Теоритические сведения}

В ходе выполения задания для произведения необходимых вычислений и преобразований использовались операции деление "\textbackslash" и деление с остатком "\%". Также использовались циклы for, while и конструкция if...else. Кроме того, были применены функции стандартной библотеки из заголовочного файла stdio.h для ввода и вывода информации, math.h для выполнения вычислений.


\subsection{Проектирование}

В ходе проектирования были выделены следущая функция:

\begin{itemize}
	\item int double\_ even\_ numbers(int)

	Функция ищет число цифр в числе, проверяет цифру на четность и в случае истинности дублирует ее.
\end{itemize}


\subsection{Описание тестового стенда и методики тестирования}

Среда разработки QtCreator 3.5.0, компилятор gcc (Debian 4.9.2-10) 4.9.2, операционная система Linux version 3.16.0-4-586.

Для тестирования работы программы были выполнены статический анализ, также было проведено автоматической тестирование.

\subsection{Тестовый план и результаты тестирования}

	Для статического анализа использовалась утилита Cppcheck.
	
	\vspace{\baselineskip}
	 Cppcheck выдал незначительные предупреждения.
	
	

	\vspace{\baselineskip}
	
	В ходе автоматического тестирования  вызывалась функция double\_ even\_ numbers. Результаты тестирования предоставлены в листингах.
	 \vspace{\baselineskip}
 

\subsection{Выводы}
В ходе выполнения я отработал навыки работы с циклами.
\subsection*{Листинги}
double\_ even\_ numbers.c
\lstinputlisting[]
{../subdirprojectt/lib/double_even_numbers.c}

\vspace{\baselineskip}

ui\_ double\_ even\_ numbers.c
\lstinputlisting[]
{../subdirprojectt/app/ui_double_even_numbers.c}

\chapter{Массивы}
\section{Задание 1}
\subsection{Задание}

Удалить из массива $A(n)$ нулевые элементы, передвинув на их место следующие элементы без нарушения порядка их следования. В результате должен получиться массив меньшего размера, не содержащий нулей.

\subsection{Теоритические сведения}

Для выполнения задания использовался цикл for, конструкция if...else, а также функции стандартной библиотеки из заголовочного файла stdlib.h для динамического выделения и освобождения памяти, stdio.h для ввода, вывода информации и работы с файлами и math.h для выполнения вычислений.

\subsection{Проектирование}

Ввод и вывод данных реализован с помощью файлов. Входной файл должен содержать некоторое количество целых чисел, записанных через пробел. Выходные файлы создаются по ходу программы и также содержат целые числа, записанные через пробел.
 
В ходе проектирования были выделены следующие функции:

\begin{itemize}
 	
 	\item int array\_ not\_ zero(int*, int)
 	Функция получает массив целых чисел, считанный из файла, а так же его размер. Затем по циклу ищет нули и удаляет их с массива, затем возвращает индекс последнего элемента массива для последующего вывода.
 	
\end{itemize}
	
\subsection{Описание тестового стенда и методики тестирования}
Среда разработки QtCreator 3.5.0, компилятор gcc (Debian 4.9.2-10) 4.9.2, операционная система Linux version 3.16.0-4-586.

Для тестирования работы программы были выполнены статический анализ, также было проведено автоматической тестирование.
\subsection{Тестовый план и результаты тестирования}

Для статического анализа использовалась утилита Cppcheck.

\vspace{\baselineskip}
Cppcheck выдал незначительные предупреждения.

\subsection{Выводы}

При выполнении задания я понял принцип организации программы при работе с выделением динамической памяти, научился работать с файлами.

\subsection*{Листинги}

array\_ without\_ nulls.c
\lstinputlisting[]
{../subdirprojectt/lib/array_without_nulls.c}

\vspace{\baselineskip}

ui\_ array\_ without\_ nulls.c

\lstinputlisting[]
{../subdirprojectt/app/ui_array_without_nulls.c}

\chapter{Строки}
\section{Задание 1}
\subsection{Задание}

В русском языке, как правило, после букв Ж, Ч, Ш, Щ пишется И, А, У, а не Ы, Я, Ю. Проверить заданный текст на соблюдение этого правила и исправить ошибки.


\subsection{Теоритические сведения}

Для выполнения задания использовался цикл for, конструкция if...else, а также функции стандартной библиотеки из заголовочного файла stdlib.h для динамического выделения и освобождения памяти, stdio.h для ввода, вывода информации и работы с файлами и string.h для работы со строками. 

\subsection{Проектирование}

В ходе проектирования были выделены следующие функции:

\begin{itemize}
	 \item void check\_ sizzling(char*, int)
 	Функция получает строку и ее размер, затем в строке выполняется поиск шипящей согласной и перенаправление на проверку последующей буквы, и при необходимости ее изменение.
 	
 	\item int check\_ vowel(char)
 	Функция проверяет на ошибку следующую после шипящей букву, и возвращает соответствующее значение.
\end{itemize}
	
	
\subsection{Описание тестового стенда и методики тестирования}
Среда разработки QtCreator 3.5.0, компилятор gcc (Debian 4.9.2-10) 4.9.2, операционная система Linux version 3.16.0-4-586.

Для тестирования работы программы были выполнены статический анализ, также было проведено автоматической тестирование.
\subsection{Тестовый план и результаты тестирования}

Для статического анализа использовалась утилита Cppcheck.

\vspace{\baselineskip}
 Cppcheck не выдал ошибок.

\vspace{\baselineskip}

\subsection{Выводы}

При выполнении задания я отработал навыки работы с файлами и научился пользоваться функциями для работы со строками.
\subsection*{Листинги}
check\_ sizzling.c
\lstinputlisting[]
{../subdirprojectt/lib/check_sizzling.c}

\vspace{\baselineskip}

ui\_ check\_ sizzling.c
\lstinputlisting[]
{../subdirprojectt/app/ui_checking_sizzling.c}


\chapter{Инкапсуляция}
\section{Задание 1}
\subsection{Задание}

Реализовать класс РАЦИОНАЛЬНОЕ ЧИСЛО (представимое в виде m/n). Требуемые методы: конструктор, деструктор, копирование, сложение, вычитание, умножение, деление, преобразование к типу double.

\subsection{Теоритические сведения}

Для выполнения задания использовался цикл for, конструкция if...else, а также класс exception стнадартной бибилиотеки.

\subsection{Проектирование}


В ходе проектирования программы было решено создать класс, который называется RationalNum.
Созданный класс содержит 2 поля с модификатором доступа private:

\begin{itemize}
	\item  int numerator;
    \item  int denominator;
    Числитель и знаменатель рационального числа.
\end{itemize}
	
	В классе определен конструктор 
\begin{itemize}

	\item RarionalNum(int numerator = 1, int denominator = 8)
	
	Конструктор со значениями по умолчанию для числа. 
	
			
\end{itemize}	
В классе определены 5 методов с модификатором доступа public:		

\begin{enumerate}	
	\item void Copy(RationalNum);
	
	Метод, аналогичный конструктору копирования.
	
	\item void Sum(int);
	
	Метод обеспечивает сложение. 
	
	\item void Multi(int);
	
	Метод обеспечивает умножение.
	
	\item void Divide(int);
	
	Метод обеспечивает деление.
	
	\item double ToDouble();
	
	Метод преобразует рациональное число к типу double. Возвращает соответственно это число.
	
	Так же перегружены операторы сложения, умножения и деления.

\end{enumerate}
	Так же был создаы класса исключений:
	
		\begin{itemize}
		\item DevNull
		Исключение вызывается, когда совершается попытка деления на ноль.
		\end{itemize}
		
		
\subsection{Описание тестового стенда и методики тестирования}
Среда разработки QtCreator 3.5.0, компилятор gcc (Debian 4.9.2-10) 4.9.2, операционная система Linux version 3.16.0-4-586.

Для тестирования работы программы были выполнены статический анализ, также было проведено автоматической тестирование.

\subsection{Тестовый план и результаты тестирования}

Для статического анализа использовалась утилита Cppcheck.

\vspace{\baselineskip}
Cppcheck ошибок не обнаружил.
\vspace{\baselineskip}

Результаты автоматического тестирования представлены в листингах.

\subsection{Выводы}

При выполнении задания я понял принцип инкапсуляции и организации полей и методов класса.

\subsection*{Листинги}
rationalnum.h
\lstinputlisting[]
{../subdirprojectt/cpp-lib/rationalnum.h}

\vspace{\baselineskip}

rationalnum.cpp
\lstinputlisting[]
{../subdirprojectt/cpp-lib/rationalnum.cpp}

\chapter {Приложение}
\section{Классы для реализации заданий 1-5}


\begin{enumerate}
	\item Класс translation, перевод радиан.
	
	translation.h
	\lstinputlisting[]
	{../subdirprojectt/cpp-lib/translation.h}
	
	\vspace{\baselineskip}
	
	translation.cpp
	\lstinputlisting[]
	{../subdirprojectt/cpp-lib/translation.cpp}
	
	\item Класс tell\_ me\_ age манипуляции с возрастом.
	
	tell\_ me\_ age.h
	\lstinputlisting[]
	{../subdirprojectt/cpp-lib/tell_me_age.h}
	
	\vspace{\baselineskip}
	
	tell\_ me\_ age.cpp
	\lstinputlisting[]
	{../subdirprojectt/cpp-lib/tell_me_age.cpp}
	
	\item Класс double\_ even\_ numbers дублирование четных цифр
	
	double\_ even\_ numbers.h
	\lstinputlisting[]
	{../subdirprojectt/cpp-lib/double_even_numbers.h}
	
	\vspace{\baselineskip}
	
	double\_ even\_ numbers.cpp
	\lstinputlisting[]
	{../subdirprojectt/cpp-lib/double_even_numbers.cpp}
	
	\item Класс array для инициализации массива и удаления из него нулей.
	
	array.h
	\lstinputlisting[]
	{../subdirprojectt/cpp-lib/array.h}
	
	array.сpp
	\lstinputlisting[]
	{../subdirprojectt/cpp-lib/array.cpp}	
	
	\item 	Класс check\_ sizzling для удаления шипящих.
		
	check\_ sizzling.h
	\lstinputlisting[]
	{../subdirprojectt/cpp-lib/check_sizzling.h}
	
	check\_ sizzling.сpp
	\lstinputlisting[]
	{../subdirprojectt/cpp-lib/check_sizzling.cpp}	
	
\end{enumerate}


\section{Автоматические тесты}

	\lstinputlisting[]{../subdirprojectt/cpptest/tst_cpptesttest.cpp}
	
	
	\lstinputlisting[]{../subdirprojectt/test/tst_ilintest.cpp}

\end{document}
